\title{Notes on the partially resoleved astrometric microlensing shift with parallax}
\author{
        Peter McGill \\
                Institute of Astronony\\
                University of Cambridge\\
                pm625@ast.cam.ac.uk
}
\date{\today}

\documentclass[11pt]{article}

\usepackage{amsmath}
\usepackage{natbib}


\begin{document}
\maketitle

\begin{abstract}

The astrometric deflection model for a source being gravitationally lensed in a partially
resolved system is presented. Two regimes of this model are considered. Firstly,
the regime where the parallax effects are not present in the deflection signal is presented.
Then the regime where the parallax can be detected in the deflection signal is presented.
We conclude that if a parallax signal is detected in the deflection, the lens mass can be
directly inferred from the astrometric time series of the source. In the case where the
parallax signal is not detected then auxiliary astrometric information about the lens
is needed to constrain the lens mass.

\end{abstract}

\section{Preliminaries}
In this note, vectors are denoted $\vec{a}$, the magnitude of a vector is denoted $|a|$, 
the dot and cross products between two vectors are denoted $\vec{a}\cdot\vec{b}$ and 
$\vec{a}\times\vec{b}$ respectively.


Let a background source (the source) be located at distance $D_{S}$ from an observer, 
and a lens with mass $M$ be at distance $D_{L}$, where $0<D_{L}<D_{S}$. Let 
$\vec{\psi}_{L}$ and $\vec{\psi}_{S}$ be the angular positions of the lens and source 
respectively. We then define the dimensionless vector,
%
\begin{equation}
\vec{u}(t) = \frac{\vec{\psi}(t)}{\theta_{E}} = 
          \frac{\vec{\psi}_{L}(t)-\vec{\psi}_{S}(t)}{\theta_{E}}, 
          \quad \theta_{E} = \sqrt{\frac{4GM}{c^{2}}\frac{D_{L}-D_{S}}{D_{S}D_{L}}}.
\end{equation}
%
In the case where the lens and major source image can be resolved for the duration 
of the microlensing event, the apparent deflection of the source is given by the 
position of the major image by e.g \cite{Bramich18} as,
%
\begin{equation}
\vec{\delta}_{+}(t) = \frac{1}{2}\left(\sqrt{|\vec{u}(t)|^{2}+4}-|\vec{u}(t)|\right)
                   \theta_{E}\frac{\vec{u}(t)}{|\vec{u}(t)|}.
\label{eq:shift}
\end{equation}
%
$\vec{u}(t)$ is usually at least parameterised by the standard microlensing parameters 
$(u_{0},t_{0},t_{E})$ which we will define here, the specifics of the 
paramterisation will be addressed in the next section. $u_{0}$ is the minimum 
normalised lens source separation in the case of linear lens-source relative
motion. $t_{0}$ is the time of minimum lens-source separation again in the case 
of linear lens-source relative motion. $t_{E}$ is the Einstein time and is defined as the 
relative lens-source proper motion divided by the Einstien radius 
$|\mu_{\text{rel}}/\theta_{E}|$. In the case of linear lens-source relative motion, this
sets the timescale of the event. 
%
\section{Parameterisation of $\vec{u}(t)$}
%
Equation (2) shows that the astrometric microlensing signal is completely
defined by the direction and magnitude of $\vec{u}(t)$. In this section, we will outline
its parameterisation in two regimes. Firstly, in the regime where the motion of the observer
is negligible, then in the general case where the observer motion is included. 
In this section we will work with all vectors with the local unit north ($\hat{n}$) 
and east ($\hat{e}$) basis. Specifically we will use the notation 
$\vec{a} = (a_{n},a_{e}) = a_{n}\hat{n} + a_{e}\hat{e}$. In the following sections
I will be following the notation and conventions of \cite{Gould04}.

\subsection{Stationary observer}

Let the relative lens-source vector be $\vec{u}(t) = -(\tau(t),\beta(t))$. In the case
of linear lens-source motion and no parallax effects we have,
%
\begin{equation}
\tau(t) = \frac{t-t_{0}}{t_{E}}, \quad \beta(t) = \beta
        = u_{0}.
\end{equation}
%
This leads to $\vec{u}(t) \to \vec{u}(t;u_{0},t_{0},t_{E})$ and therefore
by equation (2) $\vec{\delta}_{+}(t) \to \vec{\delta}_{+}(t;u_{0},t_{0},t_{E},\theta_{E})$.

\subsection{Moving observer}

To include the effect of a moving observer we introduce a small perturbation to the linear
lens-source motion model. Specifically, we let the relative lens-source vector be 
$\vec{u}(t) = -(\tau(t),\beta(t))$, and with a small perturbation equation (3) becomes,
%
\begin{equation}
\tau(t) = \frac{t-t_{0}}{t_{E}}+\delta\tau(t), \quad \beta(t) =  u_{0}+\delta\beta(t).
\end{equation}
%
To find the forms of $\delta\tau(t)$ and $\delta\beta(t)$ let us consider the following.
Let $\vec{s}(t)$ be the observer-sun vector in units of AU and in the heliocentric frame. 
Let $t_{p}$ be some fixed reference (ideally close to the event maximum $t_{0}$) as seen 
by the observer. We evaluate the derivative of $\vec{s}(t)$ at this time,
%
\begin{equation}
\vec{v}_{p} =\left. \frac{d\vec{s}(t)}{dt}\right|_{t=t_{p}}.
\end{equation}
%
The offset of the sun is then,
%
\begin{equation}
\Delta\vec{s}(t) = \vec{s}(t)-(t-t_{p})\vec{v_{p}}-\vec{s}(t_{p}).
\end{equation}
%
Consider now observations towards the event at some time with given
celestial coordinates,
%
\begin{equation}
(s_{n}(t),s_{e}(t)) = (\Delta\vec{s}(t)\cdot\hat{n},\Delta\vec{s}(t)\cdot\hat{e})
\end{equation}
%
We can now write the form of $\delta\tau(t)$ and $\delta\beta(t))$ as,
%
\begin{equation}
(\delta\tau(t),\delta\beta(t)) = |\vec{\pi}_{E}|\Delta\vec{s}(t) = 
(\vec{\pi}_{E}\cdot\Delta\vec{s}(t),\vec{\pi}_{E}\times\Delta\vec{s}(t)),
\end{equation}
%
or more explicitly,
%
\begin{equation}
(\delta\tau(t),\delta\beta(t)) =  (s_{n}(t)\pi_{EN}+s_{e}(t)\pi_{EE},
                                  -s_{n}(t)\pi_{E,E}+s_{e}(t)\pi_{EN})
\end{equation}
%
where here we have introduced the microlensing parallax vector 
$\vec{\pi}_{E} = (\pi_{EN},\pi_{EE})$. It is noted that,
%
\begin{equation}
|\vec{\pi}_{E}| = \frac{\pi_{\text{rel}}}{\theta_{E}}.
\end{equation}
%
Here, $\pi_{\text{rel}}$ is the relative lens-source parallax. This leads to 
$\vec{u}(t) \to \vec{u}(t;u_{0},t_{0},t_{E},\pi_{EN},\pi_{EE})$ and therefore
by equation (2) $\vec{\delta}_{+}(t) \to \vec{\delta}_{+}(t;u_{0},t_{0},t_{E},\pi_{EN},\pi_{EE},\theta_{E})$.
Note that if you take $\pi_{\text{rel}}=(1/D_{S})-(1/D_{L})$, then the mass of the lens can be directly
obtained through $\theta_{E}$ and $|\vec{\pi_{E}}|$.

\section{Deflection model of a lensed source}

Now we write the full astrometric model for a source with a microlensing deflection term in both regimes.
In both cases we assume the source has some initial reference position, 
$\vec{\xi}_{\text{ref}}=(\xi_{\text{ref,N}},\xi_{\text{ref,E}})$ at reference time $t_{\text{ref}}$, 
and proper motion $\vec{\mu}=(\mu_{N},\mu_{E})$, and the parallax vector of the source is $\vec{\Pi}_{S}$, with the
source parallax amplitude as $\pi_{S}$. We can write the apparent position of the 
source at some time t for the stationary observer model as,
%
\begin{equation}
\begin{aligned}
&\vec{\xi}(t;\xi_{N},\xi_{E},\mu_{N},\mu_{E},t_{0},t_{E},u_{0},\pi_{S},\theta_{E}) = \\
&+\vec{\xi}_{\text{ref}} \\
&+\vec{\mu}(t-t_{\text{ref}}) \\
&+\vec{\Pi}_{S}(t;\pi_{s})\\
&+\vec{\delta}_{+}(t;t_{0},t_{E},u_{0},\theta_{E}).
\end{aligned}
\end{equation}
%
For the moving observer regime we can write the apparent position as, 
\begin{equation}
\begin{aligned}
&\vec{\xi}(t;\xi_{N},\xi_{E},\mu_{N},\mu_{E},t_{0},t_{E},u_{0},\pi_{EN},\pi_{EE},\pi_{S},\theta_{E}) = \\
&+\vec{\xi}_{\text{ref}} \\
&+\vec{\mu}(t-t_{\text{ref}}) \\
&+\vec{\Pi}_{S}(t;\pi_{s})\\
&+\vec{\delta}_{+}(t;t_{0},t_{E},u_{0},\pi_{EN},\pi_{EE},\theta_{E}).
\end{aligned}
\end{equation}

Note that if a parallax signal is detected in the deflection (we can fit the moving observer model)
then the mass of the lens can be directly inferred from the astrometric time series. However, if we don't observe
a parallax signal in the deflection (we are only able to fit the stationary observer model) then only $\theta_{E}$
can be inferred from the astrometric time series. Determination of the lens mass would require auxiliary information
about the lens (the lens parallax) which could be obtained from Gaia Data Release 2 for example.

\bibliographystyle{plainnat} 
\bibliography{microlensingMaths}

\end{document}
